%% Abstract (configurado para língua inglesa)
\begin{resumo}[Abstract]			% Título do Resumo (Abstract = Resumo em inglês)
\begin{otherlanguage*}{english}		% Língua do texto
Social entrepreneurs in Brazil historically have had great difficulties in finding the necessary funding to exert to the fully extent the activities they need to and so frequently have utilized of volunteer work campaigns and donations of food or used objects. Although the cost reductions that volunteer work and these kinds of donations are undeniable so is the necessity of money for costs like electricity, water and medical supplies. Campaigns of financial collection of smaller organizations are usually discreet, such as asking for the change of a purchase as a small donation.

In order to solve these problems there are solutions such as seeking investors, microcredits with community banks, media campaigns seeking donations and, more recently with the growth of this genre in Brazil, crowdfunding. The later one shows great potential for collection. Crowdfunding campaigns make use of social networks and communities over the Internet to acquire a huge number of small donations which end up in significant sums.

This work presents a solution in the form an online tool for social crowdfunding called Ajuda.Ai inspired in part by the Kiva \cite{flannery2007kiva} service, with an open source code and available for small and medium organizations with minimal cost to enable a virtual, secure, direct and comfortable channel to reach donors and for donors to make giving to these organizations as easy as an online purchase.

\vspace{\onelineskip}
\noindent
\textbf{Key-words}: online payments. rest api. crowdfunding. social impact. social entrepreneurship.
\end{otherlanguage*}
\end{resumo}

%% Exemplo de resumo em francês
%\begin{resumo}[Résumé]
% \begin{otherlanguage*}{french}
%    Il s'agit d'un résumé en français.
% 
%   \textbf{Mots-clés}: latex. abntex. publication de textes.
% \end{otherlanguage*}
%\end{resumo}

%% Exemplo de resumo em Espanhol
%\begin{resumo}[Resumen]
% \begin{otherlanguage*}{spanish}
%   Este es el resumen en español.
%  
%   \textbf{Palabras clave}: latex. abntex. publicación de textos.
% \end{otherlanguage*}
%\end{resumo}
% ---