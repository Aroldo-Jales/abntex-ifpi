%% ---
%% PACOTES
%% ---

%% Fontes
%% AVISO: Todas essas fontes são *bastante semelhantes* aos nomes com as quais as descrevo. Entenda: são iguais, só que oficialmente com outro nome.

%% Latin Modern (fonte padrão do LaTeX, Computer Modern, mas com suporte a caracteres acentuados)
%% Considerada a mais clássica e bonita
%\usepackage{lmodern}

%% Times
%% Considerada a mais confortável de ler quando impresso
\usepackage{mathptmx}
%% Variação da mesma fonte, com minúsculas diferenças entre uma e outra (coisas bastante técnicas como kerning, aliasing e afins) - Essa tem revisões frequentes
%\usepackage{newtxtext} \usepackage{newtxmath}

%% Arial
%% Considerada mais confortável de ler num computador
%% Oficialmente recomendada pelo manual de formatação do IFPI
%\usepackage{helvet} \renewcommand{\familydefault}{\sfdefault}

%% Palatino
%% Uma opção mais elegante à Times
%\usepackage{newpxtext}

%% Kepler
%% Variação evoluída da Palatino, com várias pequenas diferenças e refinamentos
%\usepackage{kpfonts}

%% Libertine
%% Uma fonte estilo Serif comum no Linux
%\usepackage{libertine} %\usepackage[libertine]{newtxmath}

%%%%%%%%%%%%%%%%%%%%%%%%%%%%%%%%%%%%%%%%%%%%%%%%%%%%%%%%%

\usepackage{courier}                    % Permite a utilização da fonte Courier (para códigos-fonte)
\usepackage[T1]{fontenc}				% Seleção de códigos de fonte.
\usepackage[utf8]{inputenc}				% Codificação do documento (conversão automática dos acentos)
\usepackage{indentfirst}				% Indenta o primeiro parágrafo de cada seção.
\usepackage{nomencl} 					% Lista de símbolos
\usepackage{color}						% Controle das cores
\usepackage{graphicx}					% Inclusão de gráficos
\usepackage{float}						% Melhorias para posicionamento de imagens e tabelas
\usepackage{microtype} 					% para melhorias de justificação
\usepackage{lastpage}   		        % Dá acesso ao número da última página do documento
\usepackage{booktabs}					% Comandos para tabelas
\usepackage{verbatim}					% Texto é interpretado como escrito no documento
\usepackage{multirow, array}			% Múltiplas linhas e colunas em tabelas
\usepackage[hyphenbreaks]{breakurl}		% Hifenação para URLs no texto
\usepackage[table,xcdraw]{xcolor}       % Cores para algumas tabelas especiais
\usepackage[num]{abntex2cite}				% Citações numéricas
\citebrackets[]							% Coloca os números entre colchetes ao invés de parenteses
\usepackage{abntex-ifpi/abntex-ifpi}	% Modificações do ABNTeX para o IFPI
\usepackage{abntex-ifpi/tikz-uml}	    % Pacote Tikz UML para criar UML no LaTeX
\usepackage[brazilian,hyperpageref]{backref}	 % Paginas com as citações na bibliografia