%%%%%%%%%%%%%%%%%%%%%%%%%%%%%%%%%%%%%%%%%%%%%%%
%% Anexo B (como o A foi retirado, esse que realmente será o A)
%%%%%%%%%%%%%%%%%%%%%%%%%%%%%%%%%%%%%%%%%%%%%%%
\chapter{Especificação de Casos de Uso} \label{anexo:b}

%%%%%%%%%%%%%%%%%%%%%%%%%%%%%%%%%%%%%%%%%%%%%%%
%% Anexo B - Caso de Uso 01
%% Aqui são usado os asteriscos para retirar essas seções/subseções do Sumário
%%%%%%%%%%%%%%%%%%%%%%%%%%%%%%%%%%%%%%%%%%%%%%%
\section*{Especificação do Caso de Uso 01 --- Doação}
\subsection*{Objetivo}
Este documento tem por objetivo descrever todos os fluxos envolvidos no Caso de Uso 01 - Doação. São listados e detalhados todos os atores, fluxos, requisitos funcionais e não-funcionais.

\subsection*{Identificação dos Atores}
Esta seção lista e descreve todos os atores envolvidos nos fluxos que compõem o Caso de Uso 01 - Doação.
\begin{lista}
  \item \textbf{Usuário}: Qualquer pessoa que utiliza o Ajuda.Ai;
  \item \textbf{Cliente}: Interface através da qual o Usuário utiliza o Ajuda.Ai.
\end{lista}

\subsection*{Identificação dos Fluxos}
Esta seção lista e descreve todos os fluxos que compõem o caso de uso.
\begin{lista}
  \item \textbf{Efetuar Doação}: Este fluxo descreve como se dá a efetuação de uma doação a uma Instituição.
\end{lista}

\subsection*{Detalhamento dos Fluxos}
\begin{lista}
  \item \textbf{Efetuar Doação}: Este fluxo especifica a ação de efetuar uma doação a uma Instituição. O usuário fornece dados ao cliente o qual, junto ao sistema, faz uma ordem de pagamento junto ao \emph{Gateway} configurado para a Instituição.
    \begin{itemize}
    \item \textbf{Atores}: Usuário, Cliente;
    \item \textbf{Pré-Condições}: Nenhuma;
    \item \textbf{Pós-Condições}: Nenhuma;
    \item \textbf{Requisitos Funcionais}: O sistema deve prover uma interface para execução da doação.
    \end{itemize}
	
    \textbf{Fluxo Básico}
    \begin{enumerate}
    \item O ator Cliente mostra ao ator Usuário informações sobre uma Instituição;
    \item O Usuário decide efetuar uma doação a Instituição;
    \item O Cliente solicita os dados para doação: Valor, Se o nome de quem está doando pode ou não ser publicado, Se os custos operacionais do \emph{Gateway} devem ser embutidos no valor da doação, E-mail e Nome;
    \item O Usuário informa os dados solicitados;
    \item O Cliente envia os dados ao Sistema;
    \item O Sistema valida os dados;
    \item O Sistema registra a ordem de pagamento no SGBD;
    \item O Sistema retorna ao Cliente os dados do pagamento a ser efetuado pelo Usuário;
    \item O Cliente direciona o Usuário ao \emph{Gateway} para continuar o processo de pagamento;
    \item O caso de uso se encerra.
    \end{enumerate}
    
    \textbf{Fluxo Alternativo A} \\
    No Passo 5, caso exista algum erro de comunicação entre Cliente e Sistema:
    \begin{enumerate}
    \item Uma mensagem de erro é exibida ao Usuário informando da falha;
    \item O fluxo retorna ao passo 1.
    \end{enumerate}
    
    \textbf{Fluxo Alternativo B} \\
    No Passo 6, caso exista algum erro de validação das entradas:
    \begin{enumerate}
    \item Todos os erros de validação encontrados são retornados ao Cliente;
    \item O fluxo retorna ao passo 1.
    \end{enumerate}
\end{lista}
\pagebreak

%%%%%%%%%%%%%%%%%%%%%%%%%%%%%%%%%%%%%%%%%%%%%%%
%% Anexo B - Caso de Uso 02
%%%%%%%%%%%%%%%%%%%%%%%%%%%%%%%%%%%%%%%%%%%%%%%
\section*{Especificação do Caso de Uso 02 --- Comunicação com o Doador}
\subsection*{Objetivo}
Este documento tem por objetivo descrever todos os fluxos envolvidos no Caso de Uso 02 - Comunicação com o Doador. São listados e detalhados todos os atores, fluxos, requisitos funcionais e não-funcionais.

\subsection*{Identificação dos Atores}
Esta seção lista e descreve todos os atores envolvidos nos fluxos que compõem o Caso de Uso 02 - Comunicação com o Doador.
\begin{lista}
  \item \textbf{Usuário}: Qualquer pessoa que utiliza o Ajuda.Ai;
  \item \textbf{Instituição}: Usuário representativo de uma Instituição cadastrada no Ajuda.Ai;
  \item \textbf{Cliente}: Interface através da qual o Usuário utiliza o Ajuda.Ai.
\end{lista}

\subsection*{Identificação dos Fluxos}
Esta seção lista e descreve todos os fluxos que compõem o caso de uso.
\begin{lista}
  \item \textbf{Criar \emph{Post}}: Este fluxo descreve como se dá a criação de um \emph{Post} pela Instituição;
  \item \textbf{Editar \emph{Post}}: Este fluxo descreve como se dá a alteração de um \emph{Post} pela Instituição.
\end{lista}

\subsection*{Detalhamento dos Fluxos}
\begin{lista}
  \item \textbf{Criar \emph{Post}}: Este fluxo especifica a ação de criar um \emph{post} vinculado a uma Instituição. A Instituição fornece dados ao cliente o qual, junto ao sistema, cria um \emph{post} vinculado a Instituição que está usando o Sistema no momento.
    \begin{itemize}
    \item \textbf{Atores}: Instituição, Cliente;
    \item \textbf{Pré-Condições}: Instituição cadastrada no Sistema, Instituição estar autenticada junto ao Sistema;
    \item \textbf{Pós-Condições}: \emph{Post} cadastrado;
    \item \textbf{Requisitos Funcionais}: O Sistema deve prover uma interface para criação de \emph{posts}.
    \end{itemize}
	
    \textbf{Fluxo Básico}
    \begin{enumerate}
    \item A Instituição decide criar um novo \emph{post};
    \item A Instituição navega à página de criação de Novo \emph{Post};
    \item O Cliente solicita os dados do Post: Texto da URL, Título, Subtítulo, Conteúdo (formato \emph{Markdown}), imagem do cabeçalho e se o \emph{post} será ou não publicado;
    \item A Instituição informa os dados solicitados;
    \item O Cliente envia os dados ao Sistema;
    \item O Sistema valida os dados;
    \item O Sistema registra o \emph{post} no SGBD;
    \item O Sistema retorna ao Cliente os dados do \emph{post} criado;
    \item O caso de uso se encerra.
    \end{enumerate}
    
    \textbf{Fluxo Alternativo A} \\
    Em qualquer passo, caso exista algum erro de comunicação entre Cliente e Sistema:
    \begin{enumerate}
    \item Uma mensagem de erro é exibida ao Usuário informando da falha.
    \end{enumerate}
    
    \textbf{Fluxo Alternativo B} \\
    No Passo 6, caso exista algum erro de validação das entradas:
    \begin{enumerate}
    \item Todos os erros de validação encontrados são retornados ao Cliente;
    \item O fluxo retorna ao passo 2.
    \end{enumerate}
  
  
  
  \item \textbf{Editar \emph{Post}}: Este fluxo especifica a ação de editar um \emph{post} vinculado a uma Instituição que já existe. Após escolher qual \emph{post} será alterado, a Instituição fornece dados ao cliente o qual, junto ao sistema, altera o \emph{post} vinculado a Instituição com as novas informações.
    \begin{itemize}
    \item \textbf{Atores}: Instituição, Cliente;
    \item \textbf{Pré-Condições}: Instituição cadastrada no Sistema, Instituição estar autenticada junto ao Sistema, \emph{Post} da Instituição cadastrado no Sistema;
    \item \textbf{Pós-Condições}: \emph{Post} modificado;
    \item \textbf{Requisitos Funcionais}: O Sistema deve prover uma interface para alteração de \emph{posts}.
    \end{itemize}
	
    \textbf{Fluxo Básico}
    \begin{enumerate}
    \item A Instituição decide alterar um \emph{post};
    \item A Instituição navega a uma página com uma lista de todos os seus \emph{posts};
    \item A Instituição seleciona qual de seus \emph{posts} será alterado;
    \item O Cliente solicita os dados do Post: Texto da URL, Título, Subtítulo, Conteúdo (formato \emph{Markdown}), imagem do cabeçalho e se o \emph{post} será ou não publicado;
    \item A Instituição informa os dados solicitados;
    \item O Cliente envia os dados ao Sistema;
    \item O Sistema valida os dados;
    \item O Sistema altera o \emph{post} no SGBD;
    \item O Sistema retorna ao Cliente os dados do \emph{post} alterado;
    \item O caso de uso se encerra.
    \end{enumerate}
    
    \textbf{Fluxo Alternativo A} \\
    Em qualquer passo, caso exista algum erro de comunicação entre Cliente e Sistema:
    \begin{enumerate}
    \item Uma mensagem de erro é exibida ao Usuário informando da falha.
    \end{enumerate}
    
    \textbf{Fluxo Alternativo B} \\
    No Passo 7, caso exista algum erro de validação das entradas:
    \begin{enumerate}
    \item Todos os erros de validação encontrados são retornados ao Cliente;
    \item O fluxo retorna ao passo 3.
    \end{enumerate}
\end{lista}
\pagebreak

%%%%%%%%%%%%%%%%%%%%%%%%%%%%%%%%%%%%%%%%%%%%%%%
%% Anexo B - Caso de Uso 03
%%%%%%%%%%%%%%%%%%%%%%%%%%%%%%%%%%%%%%%%%%%%%%%
\section*{Especificação do Caso de Uso 03 --- Acompanhamento das Doações}
\subsection*{Objetivo}
Este documento tem por objetivo descrever todos os fluxos envolvidos no Caso de Uso 03 - Acompanhamento das Doações. São listados e detalhados todos os atores, fluxos, requisitos funcionais e não-funcionais.

\subsection*{Identificação dos Atores}
Esta seção lista e descreve todos os atores envolvidos nos fluxos que compõem o Caso de Uso 03 - Acompanhamento das Doações.
\begin{lista}
  \item \textbf{Instituição}: Usuário cadastrado no sistema e que possua pelo menos 01 (uma) Instituição vinculada a ele;
  \item \textbf{Gateway}: \emph{Gateway} de Pagamento selecionado pela Instituição;
  \item \textbf{Cliente}: Interface através da qual o Usuário utiliza o Ajuda.Ai.
\end{lista}

\subsection*{Identificação dos Fluxos}
Esta seção lista e descreve todos os fluxos que compõem o caso de uso.
\begin{lista}
  \item \textbf{Listagem Mensal de Doações}: Este fluxo descreve como se dá a criação de uma listagem mensal de Doações recebidas pela Instituição;
  \item \textbf{Dados de uma Doação}: Este fluxo descreve como se dá a exibição de dados sobre uma Doação em Particular.
\end{lista}

\subsection*{Detalhamento dos Fluxos}
\begin{lista}
  \item \textbf{Listagem Mensal de Doações}: Este fluxo especifica a ação de listar as doações recebidas por uma Instituição em um determinado mês de um determinado ano. A Instituição fornece dados ao cliente o qual, junto ao sistema, cria uma lista, geralmente tabular, das doações recebidas.
    \begin{itemize}
    \item \textbf{Atores}: Instituição, Cliente;
    \item \textbf{Pré-Condições}: Instituição cadastrada no Sistema, Instituição estar autenticada junto ao Sistema;
    \item \textbf{Pós-Condições}: Nenhuma;
    \item \textbf{Requisitos Funcionais}: O Sistema deve prover uma interface para listagem das Doações feitas a uma Instituição.
    \end{itemize}
	
    \textbf{Fluxo Básico}
    \begin{enumerate}
    \item A Instituição necessita de uma listagem das doações de um mês;
    \item A Instituição navega a uma página para listagem de doações;
    \item O Cliente solicita os seguintes dados: De qual Instituição deve ser a lista, Mês e Ano da listagem e um dos itens de uma lista de Estados das Doações: Todos, Pago, Pronto para Receber e Cancelado;
    \item A Instituição informa os dados solicitados;
    \item O Cliente envia os dados ao Sistema;
    \item O Sistema valida os dados;
    \item O Sistema retorna ao Cliente os dados da listagem requisitada;
    \item O caso de uso se encerra.
    \end{enumerate}
    
    \textbf{Fluxo Alternativo A} \\
    Em qualquer passo, caso exista algum erro de comunicação entre Cliente e Sistema:
    \begin{enumerate}
    \item Uma mensagem de erro é exibida ao Usuário informando da falha.
    \end{enumerate}
    
    \textbf{Fluxo Alternativo B} \\
    No Passo 6, caso exista algum erro de validação das entradas:
    \begin{enumerate}
    \item Todos os erros de validação encontrados são retornados ao Cliente;
    \item O fluxo retorna ao passo 2.
    \end{enumerate}
  
  
  
  \item \textbf{Dados de uma Doação}: Este fluxo especifica a ação de ver dados de uma única doação específica. Após escolher qual doação será exibida, a Instituição fornece os dados necessários ao cliente o qual, junto ao sistema, pega e exibe as informações requeridas.
    \begin{itemize}
    \item \textbf{Atores}: Instituição, Cliente;
    \item \textbf{Pré-Condições}: Instituição cadastrada no Sistema, Instituição estar autenticada junto ao Sistema, Haverem Doações feitas a Instituição;
    \item \textbf{Pós-Condições}: Nenhuma;
    \item \textbf{Requisitos Funcionais}: O Sistema deve prover uma interface para expor dados de um pagamento.
    \end{itemize}
	
    \textbf{Fluxo Básico}
    \begin{enumerate}
    \item A Instituição decide ver dados de uma doação;
    \item A Instituição seleciona qual doação será exibida;
    \item A Instituição navega a uma página para exibição das informações da doação;
    \item O Cliente requisita ao Sistema os dados da doação;
    \item O Sistema busca os dados da doação especificada;
    \item O Sistema retorna ao Cliente os dados da doação;
    \item O caso de uso se encerra.
    \end{enumerate}
    
    \textbf{Fluxo Alternativo A} \\
    Em qualquer passo, caso exista algum erro de comunicação entre Cliente e Sistema:
    \begin{enumerate}
    \item Uma mensagem de erro é exibida ao Usuário informando da falha.
    \end{enumerate}
    
    \textbf{Fluxo Alternativo B} \\
    No Passo 5, caso a doação esteja marcada como Anônima (\codigo{anonymous = true}):
    \begin{enumerate}
    \item Todas as informações de identificação do doador são retiradas do objeto (\codigo{payeeName} e \codigo{payeeEmail});
    \item O fluxo continua ao passo 6.
    \end{enumerate}
\end{lista}

%%%%%%%%%%%%%%%%%%%%%%%%%%%%%%%%%%%%%%%%%%%%%%%%%%%%%%%%%%%%%%%