% ----------------------------------------------------------
% Trabalhos Futuros
% ----------------------------------------------------------
\chapter{Trabalhos Futuros}

\citeauthor{belleflamme2010} discute algumas conclusões teóricas sobre porque organizações sem fins lucrativos tendem a ter mais sucesso utilizando \emph{crowdfunding} examinando literatura sobre Teoria de Falha Contratual. Essa teoria se baseia no limite das motivações financeiras dos donos do negócio através de medidas como proibição ou limite dos dividendo de lucros ou doação compulsória dos lucros para projetos sociais. No contexto de empreendimentos sociais, \citeauthor{lehner2013crowdfunding} argumenta que isso pode ser visto como um forte sinal que a organização não visa lucro e pode convidar outras formas de participação dos apoiadores como, por exemplo, trabalho voluntário.

Neste sentido, se planeja a implementação de uma espécie de limite mensal para repasse financeiro onde o excedente é guardado para o mês seguinte. Por exemplo, uma meta de R\$ 1.000,00 mensais é proposta. Considerando que R\$ 1.250,00 tenha sido arrecadado ao final do mês apenas R\$ 1.000,00 seria repassado a organização e os R\$ 250,00 restantes seriam inclusos no mês seguinte, que começaria com R\$ 750,00 restantes para alcançar a meta.

\textbf{--- Daqui pra baixo é \textit{work in progress} ---}

Um dos planos futuros para a ferramenta é dar a opção ao usuário para que seja feita doações recorrentes. Esse modelo é utilizado por plataformas como Patreon\footnote{https://www.patreon.com} para apoio, em sua maioria, a projetos artísticos. Essa forma de doação é vantajosa para as Instituições uma vez que o montante mensal fixo oriundo das doações pode ser considerado como fluxo de caixa e também para o doador, pois a automação do processo de doação pode manter a ajuda permanente sem que o doador precise dedicar tempo a isso todos os meses.

Outra medida importante planejada para o sistema é a prestação de contas por parte da Instituição.

P.S.: Falta implementar o Caso de Uso 03.

\textbf{\textit{--- O que mais colocar? Como melhorar? ---}}