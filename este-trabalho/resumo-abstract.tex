%% Resumo
\begin{resumo}
Empreendimentos sociais no Brasil historicamente têm bastante dificuldade em buscar o financiamento necessário para exercer de forma plena as atividades que necessitam e frequentemente se utilizam de campanhas de trabalho voluntário e doações de alimentos ou objetos usados. Apesar da redução de custos que trabalho voluntário e doações trazem é inegável a necessidade de dinheiro para custos como eletricidade, água e medicamentos. Campanhas de arrecadação financeira de organizações menores são comumente discretas, como pedir o troco de uma compra como uma pequena doação.

Para resolver esses problemas há soluções como buscar investidores, micro crédito junto a bancos comunitários, campanhas em mídias por doações e, mais recentemente e com o crescimento dessa modalidade no Brasil, \emph{crowdfunding} (financiamento coletivo). Esta última demonstra bastante potencial às instituições graças ao crescimento do acesso a Internet, baixa relação custo/benefício e potencial de arrecadação. Campanhas de \emph{crowdfunding} se utilizam de redes sociais e comunidades na Internet para arrecadar uma grande quantidade de pequenas doações as quais resultam em somas bastante significativas.

Este trabalho apresenta uma solução na forma de ferramenta online para \emph{crowdfunding} social chamada Ajuda.Ai inspirada em parte pelo serviço Kiva \cite{flannery2007kiva}, de código-fonte aberto e disponível para pequenas e médias organizações com custo mínimo a fim que elas possam ter um canal virtual, seguro, direto e cômodo para chegar a doadores e para os doadores para tornar o ato de doar a essas organizações tão simples quanto uma compra online.

\vspace{\onelineskip}
\noindent
\textbf{Palavras-chaves}: financiamento coletivo. impacto social. empreendimento social.
\end{resumo}

%% Abstract (configurado para língua inglesa)
\begin{resumo}[Abstract]
\begin{otherlanguage*}{english}
Social entrepreneurs in Brazil historically have had great difficulties in finding the necessary funding to exert to the fully extent the activities they need to and so frequently have utilized of volunteer work campaigns and donations of food or used objects. Although the cost reductions that volunteer work and these kinds of donations are undeniable so is the necessity of money for costs like electricity, water and medical supplies. Campaigns of financial collection of smaller organizations are usually discreet, such as asking for the change of a purchase as a small donation.

In order to solve these problems there are solutions such as seeking investors, microcredits with community banks, media campaigns seeking donations and, more recently with the growth of this genre in Brazil, crowdfunding. The later shows great potential for collection. Crowdfunding campaigns make use of social networks and communities over the Internet to acquire a huge number of small donations which end up in significant sums.

This work presents a solution in the form an online tool for social crowdfunding called Ajuda.Ai inspired in part by the Kiva \cite{flannery2007kiva} service, with an open source code and available for small and medium organizations with minimal cost to enable a virtual, secure, direct and comfortable channel to reach donors and for donors to make giving to these organizations as easy as an online purchase.

\vspace{\onelineskip}
\noindent
\textbf{Key-words}: crowdfunding. social impact. social entrepreneurship.
\end{otherlanguage*}
\end{resumo}