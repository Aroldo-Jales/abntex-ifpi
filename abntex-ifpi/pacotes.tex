% ---
% PACOTES
% ---

% ---
% Pacotes fundamentais 
% ---
%\usepackage{lmodern}					% Usa a fonte Latin Modern (mais clássica)
\usepackage{mathptmx}					% Usa a fonte Times New Roman (requerida pelo IFPI)
\usepackage[T1]{fontenc}				% Selecao de codigos de fonte.
\usepackage[utf8]{inputenc}				% Codificacao do documento (conversão automática dos acentos)
\usepackage{indentfirst}				% Indenta o primeiro parágrafo de cada seção.
\usepackage{nomencl} 					% Lista de simbolos
\usepackage{color}						% Controle das cores
\usepackage{graphicx}					% Inclusão de gráficos
\usepackage{microtype} 					% para melhorias de justificação
\usepackage{lastpage}   		        % Dá acesso ao número da última página do documento
\usepackage{booktabs}					% Comandos para tabelas
\usepackage{verbatim}					% Texto é interpretado como escrito no documento
\usepackage{multirow, array}			% Múltiplas linhas e colunas em tabelas
\usepackage[hyphenbreaks]{breakurl}
\usepackage[table,xcdraw]{xcolor}       % Cores para algumas tabelas especiais
\usepackage[num]{abntex2cite}			% Citações numéricas
\citebrackets[]							% Coloca os números entre colchetes ao invés de parenteses
\usepackage{abntex-ifpi/abntex-ifpi}	% Modificações do ABNTeX para o IFPI
\usepackage{abntex-ifpi/tikz-uml}	    % Pacote Tikz UML para criar UML no LaTeX
% ---
		
% ---
% Pacotes de citações
% ---
\usepackage[brazilian,hyperpageref]{backref}	 % Paginas com as citações na bibliografia
%\usepackage[alf]{abntex2cite}	                 % Citações padrão ABNT
% ---