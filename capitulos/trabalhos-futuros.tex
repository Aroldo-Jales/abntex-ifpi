% ----------------------------------------------------------
% Trabalhos Futuros
% ----------------------------------------------------------
\chapter{Trabalhos Futuros}

\citeauthor{belleflamme2010} discute algumas conclusões teóricas sobre porque organizações sem fins lucrativos tendem a ter mais sucesso utilizando crowdfunding examinando literatura sobre Teoria de Falha Contratual. Essa teoria se baseia no limite das motivações financeiras dos donos do negócio através de medidas como proibição ou limite dos dividendo de lucros ou doação compulsória dos lucros para projetos sociais.

No contexto de empreendimentos sociais \citeauthor{lehner2013crowdfunding} argumenta que isso pode ser visto como um forte sinal que a organização não visa lucro e pode convida outras formas de participação dos apoiadores como, por exemplo, trabalho voluntário.

Neste sentido, o autor propõe a implementação de uma espécie de limite mensal para repasse financeiro onde o excedente é guardado para o mês seguinte. Por exemplo, uma meta de R\$ 1.000,00 mensais é proposta. Considerando que R\$ 1.250,00 tenha sido arrecadado ao final do mês apenas R\$ 1.000,00 seria repassado a organização e os R\$ 250,00 restantes seriam inclusos no mês seguinte, que começaria com R\$ 750,00 restantes para alcançar a meta.