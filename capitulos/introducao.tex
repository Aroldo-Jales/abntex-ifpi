% ----------------------------------------------------------
% Introdução
% ----------------------------------------------------------
\chapter{Introdução}

Nesse capítulo será feita uma contextualização sobre \emph{Crowdfunding} ou Financiamento Coletivo, de onde ele surgiu, evoluiu e como este mercado se apresenta no Brasil. Em seguida é apresentado uma justificativa de como é possível utilizar desse fenômeno para financiamento de iniciativas sociais.

\section{Contextualização}
Em sua essência, Financiamento Coletivo é parte de um conceito mais amplo chamado Contribuição Colaborativa (ou Colaboração Coletiva - do inglês \emph{Crowdsourcing}). Em plataformas dessa modalidade de colaboração utiliza-se do "coletivo" para se obter ideias, \emph{feedback} e soluções para problemas através de uma chamada ampla via Internet e a custo zero, ou bastante reduzidos.

Sites como \emph{The Mechanical Turk}\footnote{https://www.mturk.com} oferecem uma plataforma onde pessoas ou organizações podem colocar pedidos de micro-trabalhos, como votar na qualidade de traduções ou classificar vídeos em relação a seu conteúdo. Como recompensa a pessoa que faz essas tarefas recebe micro-pagamentos de, por exemplo, U\$ 0,15 (quinze centavos de dólar) por vídeo classificado. Outros como o \emph{Kickstarter}\footnote{https://www.kickstarter.com} oferecem uma plataforma e rede social para arrecadar fundos para projetos normalmente relacionados a artes audiovisuais e atualmente é uma das mais prominentes plataformas de \emph{crowdfunding}. Em Novembro de 2016, nove das dez mais bem financiadas campanhas de \emph{crowdfunding} foram feitas via Kickstarter\footnote{Consultado em 31 jan. 2017 em http://crowdfundingblog.com/most-successful-crowdfunding-projects/}. Juntas essas campanhas arrecadaram mais de U\$ 102 milhões.

Além de casos consolidados como esses, grandes empresas da Internet como Google, Amazon e Facebook estão apostando cada vez mais em soluções de \emph{Crowdsourcing} para inúmeras tarefas que necessitam de interação humana e para treinamento de Inteligências Artificiais. Um caso recente é o aplicativo Android Google \emph{Crowdsourcing} \cite{cnet-google-crowdsourcing} que dá aos usuários pequenas tarefas para auxiliar o aprendizado das inteligências artificiais por trás dos produtos Google. Tarefas como reconhecimento de escrita, interpretação de textos em fotografias, avaliação de traduções e várias outras estão disponíveis para serem resolvidas através do aplicativo e treinarem as várias inteligências artificiais por trás dos produtos da empresa.

Dessa nova forma de colaboração surgiu, com o passar do tempo, um novo fenômeno: o Financiamento Colaborativo, ou \emph{Crowdfunding}. Ambos utilizam o poder de várias pessoas engajadas e pequenas contribuições de um grande número de pessoas para atingirem seus objetivos \cite{crowdfunding-culture}. Em contraste ao \emph{Crowdsourcing}, onde as pessoas financiam a si mesmas ou são financiadas com recompensas para fazerem o trabalho, o fluxo de capital no \emph{Crowdfunding} é o contrário. Projetos que utilizam \emph{crowdfunding} pedem, através de plataformas online, pequenas contribuições financeiras de contribuidores individuais ou mesmo de investidores. O objetivo desses projetos normalmente é fazer algo mais pessoal como produção artística ou apoiar a produção de softwares como, por exemplo, jogos de forma mais independente. Assim é criado um novo modelo de investimento que circunver formas tradicionais de investimento como empréstimos junto a bancos, \emph{venture capital} (capital de risco) e afins \cite{belleflamme2010}.

A primeira plataforma de \emph{crowdfunding} a ter sucesso e ser responsável por iniciar de fato o mercado de \emph{crowdfunding} nos Estados Unidos foi o \emph{ArtistShare}\footnote{http://www.artistshare.com/} em 2003 \cite{freedman2015brief}. De autoria de Brian Camelio, um músico e programador de Boston, o \emph{ArtistShare} foi um \emph{website} onde músicos podiam buscar doações de seus fãs para possibilitar aos artistas a gravação e produção digital de música. Eventualmente o site se tornou uma plataforma de financiamento coletivo para artistas audiovisuais, fotógrafos e músicos.

Seguindo essa tendencia, em 2005, Matt e Jessica Flannery idealizaram e lançaram o que é considerada a primeira plataforma para \emph{crowdfunding} social: Kiva\footnote{http://www.kiva.org}, uma agência de financiamento que utiliza \emph{crowdfunding} para prover micro crédito a empreendedores pobres em países em desenvolvimento, mais notavelmente no Leste da África, India e Ásia Central. O Kiva é uma empresa sem fins lucrativos (\emph{non-profit}) e plataforma tecnológica que liga pessoas que mesmo com pouco dinheiro, possuem a vontade de ajudar a pessoas que necessitam de recursos para ter um mínimo de qualidade de vida ou oportunidade \cite{flannery2007kiva}. A ideia começou quando os criadores do Kiva, durante uma viagem a África, conheceram o dono de uma peixaria na Etiópia que não tinha como melhorar seus lucros, pois não tinha dinheiro para comprar uma passagem de ônibus e dependia de um atravessador para comprar peixes.

No Brasil o mercado de \emph{Crowdfunding} está ainda em seu começo, mas cresce a cada ano. Como exposto por \citeauthor{globo-financiamento}, depois de passar mais de cinco anos procurando sem sucesso por investidores dispostos a financiar sua ideia, o arquiteto Márcio Cerqueira resolveu apostar em financiamento coletivo. A decisão foi fundamental para tirar do papel o Mola, espécie de \emph{Lego} que ajuda estudantes de arquitetura a entender melhor as estruturas de edifícios. O objetivo inicial era de levantar R\$ 50 mil, mas o projeto teve mais de 1.500 apoiadores e acabou arrecadando R\$ 600 mil, mais de 10 vezes a meta inicial, algo bastante incomum quando se busca financiamento com grandes investidores. Já no projeto Catarse\footnote{https://www.catarse.me}, uma das maiores plataformas nacionais de \emph{crowdfunding}, o volume arrecadado em 2016 foi de R\$ 16.2 milhões, um crescimento de 41\% em relação a 2015 \cite{catarse-retrospectiva2016}, e 134.827 pessoas apoiaram projetos na plataforma e desses 77.98\% apoiaram pela primeira vez (105.150 pessoas).

Diante do cenário de crescimento da modalidade de \emph{crowdfunding} no Brasil e da possibilidade de se utilizar dessa modalidade para financiamento de projetos sociais, uma ferramenta para fazer isso se faz tanto oportuna como necessária.



\section{Justificativa}
A situação atual das ONGs\footnote{Organizações Não-Governamentais}, como normalmente são chamadas organizações sem fins lucrativos, inclui dificuldades de várias ordens, mas as mais comuns, e que muitas vezes impedem a iniciativa de continuar ou até mesmo começar são dificuldades em identificar fontes de financiamento e captar recursos. Elas enfrentam críticas sobre o papel que ocupam na economia e na sociedade, sua relação com o governo e as empresas \cite{GOUVEIA2007}. Além desses problemas, muitas vezes ONGs têm problemas para captação de recursos junto a pessoas físicas, pois normalmente seus gestores acreditam que o voluntariado é o bastante, como indicado por \citeauthor{modeloGestaoONG}. No Brasil, atualmente se vê um crescimento do trabalho voluntário a um ponto que alguns autores \cite{fagundes2012repercussoes} consideram isso como influência negativa a implementação de determinadas politicas sociais governamentais para diminuição da pobreza. Entende-se que se as próprias pessoas estão se mobilizando para resolver alguns problemas sociais, os mesmos se tornam menores e consequentemente menos recursos necessitam ser alocados para isso.

Apesar do crescimento das plataformas nacionais de \emph{crowdfunding}, da quantidade de projetos e de apoiadores, essas plataformas têm taxas de uso comumente superiores a 10\%. Parte deste percentual são custos do \emph{gateway} de pagamentos, empresas que prestam serviço de recebimento de pagamentos online, utilizado pelo serviço como, por exemplo, o MoIP\footnote{3,49\% + R\$0,69 a 5,49\% + R\$0,69 por transação - Consultado em 27/01/2017 em https://moip.com.br/tarifas/}. Além desses custos, as plataformas nacionais não disponibilizam opção de escolha em relação a qual \emph{gateway} de pagamento o projeto deseja utilizar. O Catarse, por exemplo, utiliza o \emph{gateway} Pagar.me\footnote{Consultado em 27/01/2017 em http://crowdfunding.catarse.me/nossa-taxa}. Outro grande serviço nacional, o Kickante utiliza MoIP como \emph{gateway} de pagamento\footnote{Consultado em 27/01/2017 em https://www.kickante.com.br/termos/termos-de-uso, item 9.1.2}. Em virtude dessa falta de escolha, o projeto deste trabalho traz a inovação da escolha do \emph{gateway} de pagamento utilizado a fim de permitir o uso do \emph{gateway} com a melhor relação custo/benefício à instituição a qual pode ter, por exemplo, parceria com o mesmo.

Ante os custos apresentados, as dificuldades envolvidas em outras formas de financiamento e as facilidades e potenciais benefícios, este trabalho se propõe a criar uma plataforma de \emph{crowdfunding} de código aberto chamada Ajuda.Ai que acarrete o mínimo custo possível para os projetos financiados pela plataforma, focada para organizações de pequeno porte. Para esse objetivo, um dos pontos principais e diferenciais da ferramenta é a possibilidade da escolha de qual \emph{gateway} de pagamento será usado para processar as doações. Além disso, nenhum custo fora os embutidos pelo próprio \emph{gateway} de pagamento será acrescido às doações, dando assim uma maior margem ao projeto sobre as doações recebidas.

Custos de manutenção e hospedagem do projeto serão cobertos inicialmente através de capital pessoal e, com o crescimento do Ajuda.Ai, há a possibilidade de se utilizar a própria plataforma para captação de recursos para mantê-lo ou, igualmente ao projeto Kiva, buscar investimento junto a investidores anjo ou filantropos.



\section{Objetivos}
\subsection{Objetivo Geral}
O objetivo geral deste trabalho é prover um plataforma simples para facilitação de captação de recursos financeiros via Internet em modalidade \emph{Crowdfunding} focada em organizações de menor porte. A ferramenta Ajuda.Ai irá prover uma melhora em relação às disponíveis no mercado através da possibilidade de seleção e suporte a diferentes \emph{gateways} de pagamento, isenção de taxas de uso da própria plataforma de \emph{crowdfunding}, minimizando as taxas sobre as doações e provendo uma forma simples e cômoda para os doadores alcançarem as instituições de seus interesses.



\subsection{Objetivos Específicos}
\begin{lista}
  \item Traçar estratégia de oferta de redução de custos através de opções de \emph{gateways} de pagamento;
  \item Definir arquitetura de software que privilegie redução de custos de hospedagem e manutenção;
  \item Explanar a arquitetura básica de \emph{gateways} de pagamento e as possibilidades de integração com tal arquitetura.
\end{lista}



\section*{Resumo}
Neste capítulo foi apresentada uma contextualização sobre o problema tratado neste trabalho e a justificativa de tal assunto, que pode ser resumida como sendo necessidade de uma alternativa moderna e simplificada para captação de recursos para ONGs. Ao final, foram detalhados os objetivos gerais e específicos do trabalho.

Os próximos capítulos estão organizados da seguinte forma:

\begin{lista}
  \item \textbf{Fundamentação Teórica:} Neste capítulo são apresentados todos os conceitos teóricos utilizados no desenvolvimento da solução proposta no presente trabalho;
  \item \textbf{Ajuda.Ai:} Capítulo dedicado a apresentação da solução implementada, detalhando funcionalidades, utilização e demais detalhes da implementação;
  \item \textbf{Conclusão:} Neste capítulo é feita a conclusão do trabalho dado seus objetivos propostos e são listados os trabalhos futuros para melhorar e/ou expandir a utilização da solução.
\end{lista}