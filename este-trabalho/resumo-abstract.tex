%% Resumo
\begin{resumo}
Empreendimentos sociais no Brasil historicamente têm bastante dificuldade em buscar o financiamento necessário para exercer de forma plena as atividades que necessitam e frequentemente utilizam-se de campanhas de trabalho voluntário e doações de alimentos ou objetos usados. Apesar da redução de custos que trabalho voluntário e doações trazem é inegável a necessidade de dinheiro para custos como eletricidade, água e medicamentos. Campanhas de arrecadação financeira de organizações menores são comumente discretas, como pedir o troco de uma compra como uma pequena doação.

Para resolver esses problemas há soluções como buscar investidores, micro crédito junto a bancos comunitários, campanhas em mídias por doações e, mais recentemente e com o crescimento dessa modalidade no Brasil, \emph{crowdfunding} (financiamento coletivo). Esta última demonstra bastante potencial às instituições graças ao crescimento do acesso a Internet, baixa relação custo/benefício e potencial de arrecadação.  Este trabalho apresenta uma solução na forma de ferramenta online para \emph{crowdfunding} social inspirada em parte pelo serviço Kiva \cite{flannery2007kiva} e disponível para pequenas e médias organizações a fim que elas possam ter um canal virtual, seguro, direto e cômodo para chegar a doadores e para os doadores para tornar o ato de doar a essas organizações tão simples quanto uma compra online.

  \vspace{\onelineskip}
  \noindent
  \textbf{Palavras-chaves}: financiamento coletivo. impacto social. empreendimento social.
\end{resumo}

%% Abstract (configurado para língua inglesa)
\begin{resumo}[Abstract]
	\begin{otherlanguage*}{english}
		Abstract do artigo. Em inglês obviamente. Deixa pra depois hehehe
        
		\vspace{\onelineskip}
		\noindent
		\textbf{Key-words}: crowdfunding. social impact. social entrepreneurship.
	\end{otherlanguage*}
\end{resumo}